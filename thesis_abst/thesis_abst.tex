\documentclass[10pt]{jarticle}
\usepackage{float}
\usepackage{adrobo_abst}
\usepackage[dvipdfmx]{graphicx}
\usepackage{amssymb,amsmath}
\usepackage{bm}
\usepackage[superscript]{cite}
\usepackage{enumerate}
\usepackage{url}
%\usepackage[absolute]{textpos}

\renewcommand\citeform[1]{(#1)}

\begin{document}
    
    \makeatletter
    \doctype{2023年度卒業論文概要}
    \title{画像認識を用いた勤怠管理システム}{}
    \etitle{Attendance management system using image recognition}{}
    
    \author{19C1004\hspace{.5zw}池田泉海}
    \eauthor{Motomi Ikeda}
    
    \makeatother
    
    \abstract{We propose a method to automate the attendance management system.
    Conventional attendance management systems require manual time stamping,
    leading to problems such as the effort involved in stamping for each entry and exit,
    and the possibility of forgetting to stamp.
    To avoid this problem, we attempt to use of image recognition to automate the time stamping process.
    The proposed method detects attendance and departure by employing image recognition to detect person.
    Automating the time stamping process has resulted in a reduction in effort for each entry and exit, and an improvement in the objectivity of the data.}
    
    \keywords{image recognition, Attendance management system}
    
    \maketitle
    
    \supervisor{指導教員:上田隆一 准教授}
    
    \section{緒\hspace{2zw}言}%===========================
    
    近年、労働環境の変化に伴い、効率的で正確な勤怠管理の重要性が増している。
    しかし、従来の勤怠管理システムでは手動での打刻が必要であり、これには出退勤毎の打刻労力や打刻忘れ等の複数の問題が存在する。
    本研究では、これらの問題に対処するため、画像認識を用いて打刻を自動化した新たな勤怠管理システムの提案を行う。
    従来の手法に比べ、このシステムは利用者にとって手間を軽減し、データの客観性を向上させることが期待される。
    本論文では、まず勤怠管理の現状に焦点を当て、次に提案手法の詳細とその効果について述べる。
    最後に、本研究の構成や進行方針について説明する。

    \section{提案手法}%===========================
     
    \section{実装}%===========================
      
    \section{結\hspace{2zw}言}%===========================

    \vspace{5truemm}
    {\footnotesize
        \begin{thebibliography}{99}
            
    %         \bibitem{工大2005}
    %         工大太郎: ``ロボットのしくみ'', 
    %         日本機械学会論文誌A, 
    %         Vol.~108, No.~1034 (2005), pp.~1--2.
            
    %         \bibitem{Shibutani2004}
    %         Y. Shibutani: ``Heinrich's Law Resulted Pattern Dynamics --Part2--'',
    %         Proceedings of the 79th Kansai Branch Regular Meeting of the Japan Society of Mechanical Engineers,  
    %         No.~04--05 (2004), pp.~205--206.
            
    %         \bibitem{Handbook1979}
    %         The Japan Society of Mechanical Engineers ed.: ``JSME Date Handbook: Heat Transfer'', 
    %         (1979), p.~123, The Japan Society of Mechanical Engineers.
            
    %         \bibitem{Kikuchi2017}
    %         K. Kikuchi, M. Miura, K. Shibata, J. Yamamura: ``Soft Landing Condition for Stair-climbing Robot with Hopping Mechanism'', 
    %         Journal of JSDE, Vol.~53, No.~8 (2018), pp.~605--614, \url{https://doi.org/10.14953/jjsde.2017.2774}.
            
    %         \bibitem{Adrobo2019}
    %         千葉工業大学 未来ロボティクス学科 学科概要: 
    %         \url{http://www.robotics.it-chiba.ac.jp/ja/subject/index.html}, 
    %         (参照日 2023年1月29日). 
            
        \end{thebibliography}
    }
    \normalsize
    
\end{document}
