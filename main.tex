\documentclass[10pt]{jarticle}
\usepackage{float}
\usepackage{adrobo_abst}
\usepackage[dvipdfmx]{graphicx}
\usepackage{amssymb,amsmath}
\usepackage{bm}
\usepackage[superscript]{cite}
\usepackage{enumerate}
\usepackage{url}
%\usepackage[absolute]{textpos}

\renewcommand\citeform[1]{(#1)}

\begin{document}
    
    \makeatletter
    \doctype{2023年度卒業論文概要}
    \title{画像認識を用いた勤怠管理システム}{}
    \etitle{Attendance management system using image recognition}{}
    
    \author{19C1004\hspace{.5zw}池田泉海}
    \eauthor{Motomi Ikeda}
    
    \makeatother
    
    \abstract{We propose a method to automate the attendance management system.
    Conventional attendance management systems require manual time stamping,
    leading to problems such as the effort involved in stamping for each entry and exit,
    and the possibility of forgetting to stamp.
    To avoid this problem, we attempt to use of image recognition to automate the time stamping process.
    The proposed method detects attendance and departure by employing image recognition to detect person.
    Automating the time stamping process has resulted in a reduction in effort for each entry and exit, and an improvement in the objectivity of the data.}
    
    \keywords{image recognition, Attendance management system}
    
    \maketitle
    
    \supervisor{指導教員:上田隆一~准教授}
    
    \section{緒\hspace{2zw}言}%===========================
近年,労働環境の変化に伴い,効率的で正確な勤怠管理の重要性が増している.
従来の勤怠管理システムでは,打刻に生態認証や顔認証を導入するなどして,
データの客観性が高められている.
しかし,従来の勤怠管理システムは,ほとんどが手動での打刻が必要であり,
これには出退勤毎の打刻労力や打刻忘れ等の問題が存在する.\par
本研究では,これらの問題に対処するためにYOLOv8\cite{Ultralitics2023}による画像認識で打刻を自動化した勤怠管理システムの提案を行う.
YOLOv8は物体検出アルゴリズムの一つで,
リアルタイムでの物体検出によく用いられている.
これはYOLOは画像を一度だけ見て物体の位置とクラスを同時に予測することで
高速な物体検出ができるためである.\par
これらの特徴を利用して,リアルタイムでカメラ画像から人物を検出することで打刻を自動化する.
打刻を自動化することで,従来の手動で打刻をする手法に比べ,利用者にとって手間を軽減し,データの客観性を向上させることができると考えられる.

    \section{提案手法}%===========================
     
    \section{実験}%===========================

    \section{評価}%===========================
      
    \section{結\hspace{2zw}言}%===========================

    \vspace{1truemm}
    {\footnotesize
        \begin{thebibliography}{99}

            \bibitem{Ultralitics2023}
            Ultralytics (2023) ultralytics [Source code]. \url{https://github.com/ultralytics/ultralytics.}
            
        \end{thebibliography}
    }
    \normalsize
    
\end{document}
